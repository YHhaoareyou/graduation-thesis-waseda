\chapter{Discussion}\label{ch:6}

% Results Interpretation and Explanation

At first, by the results in Section 5.3.3, we confirmed that users felt others' existence and interaction with them, which indicates that the socializing mechanism worked in our prototype.
The free responses also enable us to take a glance of real cases of the interaction between participants.

% motivation
By the results in Section 5.3.1, three of the components in our framework, location-based AR, co-creation and interaction between users, are all proved successed to motivate users to access the campus, where co-creation had the most influence.
Furthermore, users were motivated the most when all components combined together instead of being evaluated separately.
In addition, the largest standard deviation and free responses with regard to interaction between users indicates that participants' opinions about the component had more disagreement.

% image
By the results in Section 5.3.2, location-based AR, co-creation and interaction between users had influence on changing image of the campus more than 'Changed a little' and close to 'Somehow changed',
where interaction between users changed the most among the components.
Similar to the results of motivation, changes in image were influenced the most when all components combined together instead of being evaluated separately,
and the largest standard deviation and free responses with regard to interaction between users again indicates the variance by this component.
Among the free responses, positive ones can be categorized to 'taking more notice of different places' and 'having a more social image',
and negative ones either state less changes in image or doubt the influence of user-user interaction on image of a location.

% Answer your research question
According to the interpretation above, both location-based AR and co-creation do have positive influence on both attracting people and changing a place's image, which answers our first research question presented in Chapter 1.
Interaction between users also motivates people to access the place, answering the third research question.
The positive answers to the two research questions match our expectation and indicate the feasibility of our framework in revitalizating a place, at least in Nishi-Waseda campus.

Furthermore, corresponding to the second research question, co-creation works slightly better than location-based AR on both attracting people and changing a place's image, while the combination of the two components results in more influence than any of them.
This indicates that location-based AR and co-creation implemented together, which is how our proposed framework was designed, works better than any of them implemented alone.
In addition, the lower effect of location-based AR might be impacted by the technological restriction in our prototype. If the graffiti can be created everywhere in the campus or be created in three-dimension, users might be influenced more by the improved experience of location-based AR.

Notably, interaction between users also succeeds to affect changes in image of the place, and the effect is even higher than both location-based AR and co-creation solely, although it receives the most controversial opinions.
Some participants also described their changes in image as 'a more social one', indicating the importance of socialization in image changing.
However, we cannot assert that user-user interaction itself works better than either location-based AR or co-creation.
In the evaluation results of preference between situation with and without location-based AR, some participants expressed that they prefer the one with location-based AR due to the chances to interact with other users in reality.
Many of the free responses of location-based AR's effect on motivation also mentioned that they enjoyed sharing graffiti to other users and seeing other graffiti. 
Results in the control experiment also show lower performances of the non-AR prototype, and the free responses also mentioned their interest in interacting with other users with location-based AR.
To speak specifically, participants might be influenced the most by 'interaction based on location-based AR', instead of the interaction mechanism itself, while we did not make clear of the difference in this study.

Meanwhile, there are also opposite opinions stated that interaction happens among people and less related to a place, so it has less impact on the image of the place.
This does not conflict with our idea since we did not planned to inspect user-user interaction's effect on image changing at first, while we are still surprised by the observation of people's different attitudes toward user-user interaction's relation with the place it occurs.

% Generalization
Although the experiment was conducted in Nishi-Waseda campus, except pictures taken in the campus and the places we assigned for displaying the graffiti,
there is no feature unique to the campus in our prototype.
Therefore, we suppose our framework is also feasible for public facilities with:
\begin{quote}
  \begin{itemize}
    \item No high buildings surrounded due to GPS accuracy
      \begin{itemize}
        \item Indoor places are fine as long as the building is not tall.
      \end{itemize}
    \item No or little restriction on number of visitors so that more people can participate in co-creation
    \item Safe and comfortable places to stay for a while to use mobile devices for safety and weather issues
      \begin{itemize}
        \item For outdoor places, having a roof are prefered for rainy days, and usually just a roof does not affect GPS accuracy.
      \end{itemize}
  \end{itemize}
\end{quote}

Examples of the places may include other campuses, parks, office buildings or community centers with low heights, shopping streets where vehicles cannot enter, and so on.