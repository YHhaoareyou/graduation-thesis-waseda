\chapter{Backgrounds} \label{ch:2}
% Explain concepts required to understand this paper, with including references to existing works that introduced the concepts

\section{Pandemic's impact}

Google has been collecting their users' mobility data since the beginning of 2020 \cite{googlemobilityreports} \cite{ourworldindata_2020}.
Results indicate that people do access public places, including transit stations, workplaces and parks, less than before pandemic started spreading.
The pandemic also accelerates the process of digitalization \cite{amankwah-amoah_khan_wood_knight_2021}, which results in a decrease of people commute physically as well.
There are also investigations reporting that more than tens of thousands of store closed in Japan during the pandemic.
Other investigations reported that remote working has becoming a permanent phenomenon around the world \cite{saad_wigert_2021}.
In Japan, government even composed a policy to discourage employees to commute physically.
The above situations have resulted in more unused facilities left on the society.
The U.S. government holds about 45,000 underused or underutilized buildings according to an investigation by Harvard Business Review \cite{hounsell_2020}.

\section{Regional Revitalization}
Regional Revitalization is proposed by Japanese government, aiming at combining local unique features or specialties and new ideas or technology,
in order to stimulate rural economics to balance the gap between cities and rural areas \cite{sawaji_2019}.

Common approaches include improving quality or design of existing local products with new techniques, launching new industries with local features, and broadening promotion on SNS or other media.
Of course, a standard does not exist in the field of Regional Revitalization, and there are different cases adopting diverse approaches,
such as inserting real landscapes or local products into dramas or animations to attract audience,
or inviting artists to create graffiti at shopping streets to get their customers back \cite{中野経済新聞_2015}\cite{サンテレビニュース_2021}\cite{urbact_2019}.

As the development of Augmented Reality, there are also cases implementing Augmented Reality in their revitalization projects,
such as placing a virtual castle on a historical ruin \cite{井上道哉_長澤可也_2021} and displaying interactive digital contents beside local physical exhibits \cite{センチメンタル価値再生_2016}\cite{armarker_and_behavior_log_2011}.

\section{Location-based Augmented Reality}
Augmented Reality (AR) \cite{van_krevelen_poelman_2010} utilizes camera on smartphone or glasses to capture the landscape of real world,
and then digital contents are displayed on the captured landscape so that digital information looks combined with the reality.
Location-based Augmented Reality makes use of geographical information such as GPS data or feature points of a landscape,
so that displayed contents are located corresponding to a specific location.
Pokémon Go is one of the famous cases of Location-based AR, which displays virtual characters 'Pokémon' based on geographic coordinates around the world and requires players to move physically to catch them \cite{pokemongo_homepage}.
The game has earned more than 5 billion dollars since its launch 5 years ago \cite{strategist_2021}, indicating the enormous popularity it possesses.

Beside entertainment, Location-based AR is also applied in tourism and education cases, where the
examples include displaying educational resources on a tablet when getting close to a spot in an archaeological site \cite{law_2018},
or asking a user to challenge a quiz on one's smartphone when approaching a historical building \cite{hwang_chang_chen_chen_2017}.

\section{Co-creation}
Co-creation, in business context, is defined as a company involving its customers in the creation of products or services to suit customers' own context \cite{cocreation_definition}.
In a general context, it is also defined as any act of creativity that is shared by two or more people \cite{cocreation_definition_general}.
Co-creation can happen not merely between a company and its customers but also in occasions where value creation is conducted by ordinary people together \cite{cocreation_general_case}.
Co-creation is also studied in fields of design \cite{cocreation_definition_general}, innovation \cite{lee_olson_trimi_2012}, public sector \cite{osborne_radnor_strokosch_2016}, etc.

Our study adopts the more general definition, and we referred to researches about Co-creation in different context, which will be introduced in the next chapter.