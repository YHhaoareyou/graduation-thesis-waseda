\chapter{Backgrounds} \label{ch:2}
% Explain concepts required to understand this paper, with including references to existing works that introduced the concepts

\section{Pandemic's impact}
% Google Mobility Trends: people go to public places less than before (park, station, workplaces)
% store closed
% remote work increased
% government also encourage remote work

Google has been collecting their users' mobility data since the beginning of 2020 \cite{googlemobilityreports} \cite{ourworldindata_2020}.
Results indicate that people do access public places, including transit stations, workplaces and parks, less than before pandemic started spreading.
The pandemic also accelerate the process of digitalization \cite{amankwah-amoah_khan_wood_knight_2021}, which also resulted in a decrease of people commute physically.
There are also investigations indicating that more than tens of thousands of store closed in Japan during the pandemic.
Other investigations show that remote working has becoming a permanent phenomenon around the world \cite{saad_wigert_2021}.
In Japan, government even made a policy to discourage employees to commute physically.
The above situations resulted in more unused facilities left on the society.
The U.S. government holds about 45,000 underused or underutilized buildings according to an investigation by Harvard Business Review \cite{hounsell_2020}.

\section{Local Revitalization}
Local Revitalization is proposed by Japanese government, aiming at combining local unique features or specialties and new ideas or technology,
in order to stimulate rural economics to balance the gap between cities and rural areas.

Shopping streets attempt to attract customers back by inviting artists to create graffiti at the street.
% Definition (Japan)
% famous cases
% cases using AR
% cases using graffiti

\section{Location-based Augmented Reality}
Augmented Reality (AR) utilizes camera on smartphone or glasses to capture the landscape of real world,
and then displays digital contents on the captured landscape so as to combine digital information with reality.
Location-based Augemnted Reality makes use of geographical information such as GPS data or feature points of a landscape,
so that displayed contents are located corresponding to a specific location.
Pokemon Go is one of the famous cases of Location-based AR, which displays 'pokemons' around the world and requires players to move physically to catch them.
Beside entertainment, Location-based AR is also applied in tourism and education cases, such as displaying a 3D model of an ancient castle on its ruin in reality.

\section{Co-creation}
Co-creation involves users to participate contents creation of a service.
% definition
% cases in business
% cases with IT