\chapter{Conclusion}\label{ch:7}

In this study, we proposed a framework consisting of location-based AR, Co-creation and user-user interaction for revitalizing a general place.
We expected that the combination of location-based AR and Co-creation makes a place more attractive and brings it new images, and we also expect user-user interaction motivates people to access the place.
We developed a prototype following the framework and conducted an experiment with participants using the prototype in Nishi-Waseda campus, and the results evaluated by scales and free responses matched our expectation,
indicating that our framework is feasible for revitalization of the campus.
We then discussed its generalization to other public facilities.

Limitations of this study include the technological restriction related to location-based AR, lack of objective evaluation scales for changes in image of the place,
ambiguous evaluation results due to unclear definition of user-user interaction mechanism during the evaluation, and the concern for weather and pandemic.

The prototype in our study only allows 2D graffiti with mobile devices, although we made it to display the graffiti in different angles.
For future works adopting our proposed framework, to improve user experience, we suggest implementing a 3D sketching system like the one developed by Arora et al. \cite{arora_habib_kazi_grossman_fitzmaurice_singh_2018}, but the equipment issues should be solved first to make it usable outside the lab.
Improvement in GPS accuracy for mobile devices may also help improve the user experience.

For evaluation in future works, we suggest conducting control experiments with regard to different components in the framework to clarify their relevance between each other.
We also suggest designing a new scale for evaluating changes in the image of a general place, instead of the scale for destination image which is limited to tourism context, for more objective results.
Eventually, experiments in more diverse places are recommended in order to further evaluate the feasibility of generalizing our framework.