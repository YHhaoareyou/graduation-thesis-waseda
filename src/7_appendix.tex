\chapter*{APPENDIX A - Machine Specs} 
\begin{table*} 
\begin{center} 
\caption{Machine specs}
{\normalsize
\begin{tabular}{l c}
\toprule
Item & Value \\
\midrule
    CPU & Intel Xeon E5-2690 \\ 
    Memory & 188G \\
    OS & 18.04.5 LTS (GNU/Linux 4.15.0-121-generic x86\_64) \\
\bottomrule
\end{tabular}
} \end{center} \end{table*}


\chapter*{APPENDIX B - Derivation of the simplest form of policy gradient}
Derivation of the simplest form of policy gradient is provided below.

\begin{equation*}
    \begin{aligned}[b]
        \nabla_\theta J(\pi_\theta) 
        & = \nabla_\theta \E_{\tau \sim \pi} [R(\tau)] \\ 
        & = \nabla_\theta \int_\tau P(\tau|\theta) R(\tau)\\
        & =  \int_\tau \nabla_\theta P(\tau|\theta) R(\tau)\\
        & =  \int_\tau P(\tau|\theta) \nabla_\theta log P(\tau|\theta)R(\tau)\\
        & = \E_{\tau \sim \pi} [\nabla_\theta log P(\tau|\theta)R(\tau)] \\
        & = \E_{\tau \sim \pi} [\nabla_\theta log \pi_\theta (a_t | s_t)R(\tau)] \\
    \end{aligned}
\end{equation*}

    This is a expectation, which can be estimated with a sample mean. Denote the estimated policy gradient as $\hat{g}$: 
    $$   \hat{g} = \frac{1}{\mathit{D}} \sum_{\tau \in \mathit{D}} \sum_{t=0}^{T} \nabla_\theta log \pi_\theta (a_t | s_t)R(\tau)      $$










\addcontentsline{toc}{chapter}{APPENDIX A}  % unnumbered chapters are not added to TOC by default so adding it automatically

