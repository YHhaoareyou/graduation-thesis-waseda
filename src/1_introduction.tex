\chapter{Introduction}
% Topic

This study attempts to implement location-based Augmented Reality and Co-creation on Regional Revitalization for a campus and aims at generalization to other places.

\section{Motivations}
% Current situations
% Why there needs further research

As the pandemic of COVID-19 spreading throughout the world since 2020,
people have been forced or encouraged to stay home and restricted from accessing public places,
including tourist attractions, shops, workplaces, schools, etc.
Humans' freedom in physical space is restricted, which is accelerating the progress of digitalization.
Not only entertainment but more and more economic and even academic activities are moving online.
As the pandemic slowing down recently, despite the resumption of some physical activities,
there are places or facilities remaining desolate or unused due to financial problems,
amount of users not recovered, digitalization of activities, and so on.

Removing the unused places or facilities is a choice, but if there is an alternative that gives them new values or change people's image of them,
they can play different roles and keep contributing the society or enrich the environment.
In fact, the concept 'Regional Revitalization', which refers to the attempts to vitalize rural towns where population is falling
by making use of local speciality combined with new ideas to develop new and unique industries such as tourism, has been applied around Japan for years.
Among cases of Regional Revitalization, some of them adopt location-based Augmented Reality to help enrich the space.
Location-based Augmented Reality is defined as Augmented Reality that utilize geographical information to display contents corresponding to a physical location.
It has already used in not merely entertainment, where Pokemon GO is a famous example,
but also implemented in tourism and education, which implies its versatility and practicability.
With the application of location-based Augmented Reality and the reference of Regional Revitalization,
transformation of an unused place or facility without physical reconstruction seems to be feasible.

Current Regional Revitalization requires considering local unique specialties or features, which takes resources and time to create suitable contents,
not to mention public facilities like campuses, parks, business buildings, transport hubs which are usually lack of unique specialties or features usable for revitalization,
especially for tourism, one of the most common applications of regional revitalization.
Fortunately, these places have one property in common: users. It may be an alternative for these places to invite users back to create contents based on them,
complementing the lack of local uniqueness, attracting more users back and realize their revitalization.
We supposed that with the help of Augmented Reality, users can enjoy and create contents with less cost.
Although encouraging users back to places where they don't go anymore to create contents becomes a new problem,
we considered user-user interaction a possible solution since there are studies showing positive effects of user-user interaction on users engagement.

% Initial Inspiration

Finally, in Nishi-Waseda campus of Waseda University, the buildings are mostly white or silver, and students always describe the landscape as a factory; meanwhile students accessing the campus has become much less after the pandemic.
These two reasons became the initial inspiration for us to add more colors on our campus to make it looks more vivid as well as attract more people to come back.

\section{Objectives and importance}

% Research Questions

We composed several research questions in this study:
\begin{quote}
  \begin{enumerate}
    \item We want to examine whether location-based Augmented Reality and Co-creation can
      \begin{enumerate}
        \item make a place more attractive
        \item change a place's image for users.
      \end{enumerate}
    \item If the first question gives positive results, we want to know further about how much their influence is, both respective one and combined one.
    \item We also want to examine user-user interaction's effect on user's motivation to access a place.
  \end{enumerate}
\end{quote}

% General goals

With regard to the goals in this study, we firstly attempted to compose a framework for revitalizing the campus, and we aimed at evaluating its feasibility by answering the above research questions, for which we expected the results to be positive.
Finally we discussed the generalization of our framework to not only campus but also other public facilities or places.

% Importance of this study

As for the importance of this study, we focused on revitalization for places in general, different from current cases of Regional Revitalization that are usually applied on rural region and in tourism or education orientation.
Also, we had users comprise the contents, instead of considering specific characteristics of each location and customize the contents on the side of service provider.
Last but not least, we aimed at proposing a framework concept and evaluating it, focusing less on improving Location-based Augmented Reality in technology aspects, such as the accuracy of geographical information or object displayment.

\section{Overview of this paper}
This paper consists of 6 chapters, beginning with this chapter for introduction.
Chapter 2 explains background knowledges and concepts behind this study, including pandemic's impact, Regional Revitalization, Location-based Augmented Reality and Co-creation.
Chapter 3 introduces previous studies related to ours, and compares our work with them to make our work's importance more explicit.
Chapter 4 explains the methodology in this study, including a concept framework and the prototype we built.
Chapter 5 describes the design of evaluation and details of the user experiment, as well as displays the results.
Chapter 6 discusses the presented results, points to improve in our study, and the generalization of our proposed framework.
Chapter 7 draws a conclusion, mentions the limitation in this study, and proposes possible future works.
