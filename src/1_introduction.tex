\chapter{Introduction}
% Topic

This study attempts to implement Location-based Augmented Reality and user Co-creation on Local Revitalization for a campus and aims at generalization to other places.

\section{Motivations}
% Current situations
% Why there needs further research

As the pandemic of COVID-19 spreading throughout the world since 2020,
people were forced or encouraged to stay home and restricted from accessing public places,
including tourist attractions, shops, workplaces, schools, etc.
Humans' freedom in physical space is restricted, which accelerate the progress of digitalization.
Not only entertainment but more and more economic and even academic activities are moving online.
As the pandemic slowing down recently, despite the resumption of some physical activities,
there are places or facilities remaining unused or abandoned due to financial problems,
amount of users not recovered, digitalization of activities, and so on.

Removing the unused places or facilities is an alternative, but if it is possible to give them new values or change people's image of them,
they can play different roles and keep contributing the society or enrich the environment.
In fact, the concept 'Regional Revitalization', which referes to the attempts to vitalize rural towns where population is falling,
by making use of local speciality combined with new ideas to develop new and unique industries such as tourism, has been applied around Japan recently.
Among cases of Regional Revitalization, some of them adopt location-based Augmented Reality to help enrich the space.
Location-based Augmented Reality is defined as Augmented Reality that utilize geographical information to display contents corresponding to a physical location.
It has already used in not merely entertainment, where Pokemon GO is a famous example,
but also implemented in tourism and education, which implies its versatility and practicability.
With the application of location-based Augmented Reality and the reference of Regional Revitalization,
transformation of an unused place or facility without physical reconstruction seems to be feasible.

Current Local Revitalization requires considering local unique specialties or features, which takes resources and time to create suitable contents,
not to mention public facilities like schools, business buildings, transport hubs which are usually lack of unique specialties or features usable for revitalization,
especially for tourism cases, one of the most common applications of local revitalization.
Fortunately, these places have one property in common: users. It may be an alternative for these places to invite users back to create contents based on them,
complementing the lack of local uniqueness, attracting more users back and realize their revitalization.
We suppose that with the help of Augmented Reality, users can enjoy and create contents with less cost.
Although encouraging users back to places where they don't go anymore to create contents becomes another problem,
we consider user-user interaction a possible solution since there are works showing positive effects of user-user interaction on users engagement.

% Initial Inspiration

Finally, the buildings in our campus are mostly white or silver, and students always describe the landscape as a factory; meanwhile students accessing the campus has become much less after the pandemic.
These two reasons has become the initial inspiration for us to add more colors on our campus to make it looks more vivid as well as attract more people to come back.

\section{Objectives and importance}

% Research Questions

There are several research questions in this study:
\begin{quote}
We examine whether Location-based Augmented Reality with user co-creation does
  \begin{itemize}
    \item Make a place more attractive
    \item Change a place's image for users
    \item Form interaction between users
  \end{itemize}
\end{quote}

% General goals

In this study, firstly we aim at answering the above research questions, and we expect the results are positive.
Furthermore, we try to figure out the possibility to revitalize the campus as a response to our initial inspiration,
and generalize the concept and experience to not only campus but also other public facilities or places.

% Importance of this study

As for the importance of this study, firstly we tend to realize Local Revitalization in general, different from current cases that are usually applied on rural region and in tourism or education orientation.
Also, we let users comprise the contents, instead of considering specific characteristics of each location and customize the contents on the side of service provider.
Last but not least, we attempt to prove a possibility, focusing less on improving Location-based Augmented Reality in technology aspects like the accuracy of geographical information or object displayment.

\section{Overview of this paper}
This paper consists of 6 chapters, beginning with this chapter for introduction.
Chapter 2 explains background knowledges and concepts behind this study, including pandemic's impact, Local Revitalization, Location-based Augmented Reality and Co-creation.
Chapter 3 introduces previous studies related to ours, and compares our work with them to make our work's importance more explicit.
Chapter 4 explains the methodology in this study, including a concept model, prototype we built, and details of user experiment.
Chapter 5 conducts analysis and discussion on presented results from the user experiment.
Chapter 6 draws a conclusion, mentions limitation in this study, and proposes possible future works.
