\chapter{Related Works} \label{ch:3}
% summarize existing related / similar works, and discuss how my paper differs from them

\section{Location-based AR's effect on a place / how users view the place}
Hwang et al. developed a location-based AR learning system for supporting local culture courses.
For students who used the system in field trips, an enhancement in their local culture identity,
identification of the culture in a place where one lives, is observed \cite{hwang_chang_chen_chen_2017}.
Law created a mobile app which features a navigation map and pop-ups of educational resources when a user approaches a site physically,
and the study implicates the potential of location-based AR to enhance and disseminate the value of cultural heritage \cite{law_2018}.
These studies investigate the influence of Location-based AR on the place or on how people value the place,
while they focus more on educational goals, and their systems were developed for specific cases, which requires more knowledge and cost to implement.

Chan et al. attempted to integrate location-based AR and virtual currency to connect travelers and local shops,
form a new tourism ecosystem and further build an offline business network \cite{chan_lin_wang_lu_hsu_2017}.
In this study, the developed system is less case-specific, but the investigation is only adapted to the field of tourism business.

Therefore, we began to be curious about the influence of Location-based AR on the place or on how people value with a more general and less case-specific investigation.

\section{Location-based AR's effect on users' motivation}
Laato et al. found that a location-based AR game motivates players to go outside even during pandemic \cite{laato_islam_laine_2020}.
Lee et al. proposed a framework describing reasons of stickness to location-based AR game,
and their analysis indicates positive influences by satisfaction and sense of flow \cite{lee_chiang_hsiao_2018}.
Both of the studies chose Pokemon GO as their target to analyze how Location-based AR affects users' motivation,
while Pokemon Go's gaming features are also included in their proposed model.
Despite Pokemon GO's leading awareness among all location-based AR games,
Lee et al. pointed out that other location-based AR games also deserve investigation \cite{lee_chiang_hsiao_2018},
and we consider that an examination on not a game but a more general location-based AR service would be more representative.

Lacka's assessment indicates that full-fledged location-based AR games played in tourism destination support users to acquire knowledge about the place,
which subsequently enhances users' visit intention \cite{lacka_2018}.
Research conducted by Chan et al. mentioned above also investigated how their AR implementation motivates travelers to engage in more extensive and deeper travel experiences \cite{chan_lin_wang_lu_hsu_2017}.
Lacka focused more on tourism and learning aspects, and Chan et al. also investigated about tourism, which are the most focused fields in researches about AR recently,
and we believe that more investigations of motivation from other aspects would help location-based AR be applied in more situations.

\section{Co-creation's effect on users' motivation to access a service}

\section{User-user interaction's effect on users' engagement}

\section{Local Revitalization by Graffiti}

\section{More examples of location-based service with co-creation}
