\chapter{Related Works} \label{ch:3}
% summarize existing related / similar works, and discuss how my paper differs from them

\section{Location-based AR's effect on a place / how users view the place}
Hwang et al. developed a location-based AR learning system for supporting local culture courses.
For students who used the system in field trips, an enhancement in their local culture identity,
identification of the culture in a place where one lives, is observed \cite{hwang_chang_chen_chen_2017}.
Law created a mobile app which features a navigation map and pop-ups of educational resources when a user approaches a site physically,
and the study implicates the potential of location-based AR to enhance and disseminate the value of cultural heritage \cite{law_2018}.
These studies investigate the influence of Location-based AR on the place or on how people value the place,
while they focus more on educational goals, and their systems were developed for specific cases, which requires more knowledge and cost to implement.

Chan et al. attempted to integrate location-based AR and virtual currency to connect travelers and local shops,
form a new tourism ecosystem and further build an offline business network \cite{chan_lin_wang_lu_hsu_2017}.
The system Chan et al. developed is less case-specific, but their investigation is only adapted to the field of tourism business.

Therefore, we began to be curious about the influence of Location-based AR on the place or on how people value with a more general and less case-specific investigation.

\section{Location-based AR's effect on users' motivation}
Laato et al. found that a location-based AR game motivates players to go outside even during pandemic \cite{laato_islam_laine_2020}.
Lee et al. proposed a framework describing reasons of stickness to location-based AR game,
and their analysis indicates positive influences by satisfaction and sense of flow \cite{lee_chiang_hsiao_2018}.
Both of the studies chose Pokemon GO as their target to analyze how Location-based AR affects users' motivation,
while Pokemon Go's gaming features are also included in their proposed model.
Despite Pokemon GO's leading awareness among all location-based AR games,
Lee et al. pointed out that other location-based AR games also deserve investigation \cite{lee_chiang_hsiao_2018},
and we consider that an examination on not a game but a more general location-based AR service would be more representative.

Lacka's assessment indicates that full-fledged location-based AR games played in tourism destination support users to acquire knowledge about the place,
which subsequently enhances users' visit intention \cite{lacka_2018}.
Research conducted by Chan et al. mentioned above also investigated how their AR implementation motivates travelers to engage in more extensive and deeper travel experiences \cite{chan_lin_wang_lu_hsu_2017}.
Lacka focused more on tourism and learning aspects, and Chan et al. also investigated about tourism, which are the most focused fields in researches about AR recently,
and we believe that more investigations of motivation from other aspects would help location-based AR be applied in more situations.

\section{Co-creation's effect on a place and users' motivation}
Destination image, a term in tourism context, is defined as the aggregation of people's subjective perception,
including beliefs, ideas and impressions, associated with a destination \cite{kotler_haider_rein_2008}\cite{Lopes2011DestinationIO}.
Yilmaz's paper points out the lack of studies about how destination image occurs over time, despite destination image being studied much in tourism literature,
and the paper presents an approach to realize the formation of destination image with co-creation \cite{yilmaz_2021}.
Vries et al. built a model of antecedents of destination image co-creation and examine the effect of each antecedent \cite{glyptou_2021}.

The concept of destination image is similar with our idea about people's image of a location,
and both Vries et al. and Yilmaz's researches about destination image with co-creation indicate the potential of co-creation to influence people's image of a location in our study.

In addition, Vries et al. examined about customer engagement with Facebook brand pages, and they confirmed the influence of co-creation value on customer engagement \cite{vries_carlson_2014},
which we consider as a precedent to prove co-creation's possibility to improve users' motivation.

The studies introduced above focus specifically on tourism or business viewpoint, which provides us a room to develop our study in a broader context.

\section{User-user interaction's effect on engagement with co-creation}
Studies show that desire to contact or socializing between users motivate users to participate in co-creation activities \cite{fernandes_remelhe_2016}\cite{engstrom_elg_2015}.
Waseem et al. also found that interpersonal engagement is one of the key drivers that evoke motivations among employees to facilitate value co-creation \cite{waseem_biggemann_garry_2020}.
The influence of community in triggering users to engage in co-creation is examined as well \cite{palma_trimi_hong_2018}\cite{zhang_kandampully_bilgihan_2015}.
From these studies we confirmed that interaction between users works well on motivating people to participate in co-creation,
so we attempted to include interaction between users into our work as well to examine its effect on the context of co-creation with location-based AR.

\section{Location-based service or AR with Co-creation}
Cases of co-creation implemented in location-based services or AR application also emerged in recent years.
Anttoni Lehto et al. presented an adoption of co-creation which allowed students to initially create contents for a location-based AR learning platform \cite{lehto_lautkankare_brander_alanissila_saari_salminen_2020}.
Jorge Bacca et al. proposed a framework to utilize co-creation in designing motivational augmented reality for vocational education and training \cite{acosta_navarro_gesa_kinshuk_2019}.
Alavesa et al. developed a location-based AR client for their living labs, which is described as an environment involving users into innovation \cite{alavesa_2018}.
Leung et al. proposed a smart service network to realize co-creation of interactive dining experiences using location information \cite{leung_loo_2020}.
Slingerland et al. include users in the design of game activities to examine what kind of location-based activities citizens prefer to interact with neighbours and explore their neighbourhood \cite{slingerland_fonseca_lukosch_brazier_2020}.
With such a number of precedents, we believe that our idea, which includes implementation of location-based AR and co-creation together, is worth to be conducted and examined.
