\chapter{Experiment and Results}\label{ch:5}

\section{Experiment}
At first, we conducted a preliminary survey with 3 participants trying the prototype in Waseda University Nishi-Waseda Campus for one week.
3 participants gave us positive responses about their motivation to access campus after experiencing the prototype.
We also improved the app based on their feedbacks, such as adding features that allow users to review/edit/delete their own graffiti.
The experinemt lasted for 2 weeks. 14 males and 2 females participated,
and they are asked to use the prototype freely in the same campus at least twice a week.
Before the experiment, we asked participants about their frequencies of accessing the campus and the images of campus in their mind before and after the pandemic started spreading,
in order to understand how much impact the pandemic brought on each participant.
Instruction of using the prototype was also distributed before the experiment.
2 weeks later, after the experiment finished, participants were required to answer the questionnaires introduced in Section 5.2.

\section{Questionnaires}

\begin{quote}
  We categorized our questions into four questionnaires, each of them holding a topic:
  \begin{enumerate}
    \item Viewing location-based AR contents, the graffiti, in the campus
    \item Creating location-based AR contents, the graffiti, in the campus
    \item Interactions with other users
    \item Overall experience of using the prototype
  \end{enumerate}
\end{quote}

For each topic, we asked questions about how the experience of the topic during the experiment changes one's motivation to access the campus, image of the campus in one's mind, one's preference between the prototype or a similar one but playable at home.
For topic 1, 2 and 4, we also asked questions about participants' feeling of presence, while for topic 3 we asked questions about awareness of other users' existence and interaction.
Questions in topic 1 are prepared for measuring the effects of location-based AR, topic 2 for co-creation, topic 3 for interaction between users, topic 4 for the whole framwork.
For measuring the motivation to access the campus, we adopted Situational Motivation Scale (SIMS) \cite{guay_vallerand_blanchard_2000} in this study.

\section{Results}
\subsection{Motivations}

\subsection{Image of Campus}

\subsection{User-user Interaction}