\documentclass[a4paper, 12pt, oneside]{book}

%% packages
\usepackage[backend=biber,style=ieee]{biblatex}
\usepackage{graphicx} % for including images
\usepackage{hyperref} % for clickable embeded links (including in-doc refs)
\usepackage{minted} % for code highlighting
\usepackage{nameref} % for in-doc references | \labal{} |  \ref{}  \nameref{}
\usepackage{booktabs} % for midule
\usepackage{siunitx} % for midrule
\usepackage{amsmath,amssymb} % for expectation sign
\DeclareMathOperator{\E}{\mathbb{E}}
\usepackage[vlined, ruled]{algorithm2e}
\usepackage[table]{xcolor}
\usepackage{multirow}
\usepackage{luatexja-fontspec}

\newcolumntype{L}[1]{>{\raggedright\let\newline\\\arraybackslash\hspace{0pt}}m{#1}}
\newcolumntype{C}[1]{>{\centering\let\newline\\\arraybackslash\hspace{0pt}}m{#1}}
\newcolumntype{R}[1]{>{\raggedleft\let\newline\\\arraybackslash\hspace{0pt}}m{#1}}

%% chapter starting from 1
\setcounter{chapter}{0}

%% bibliography
\addbibresource{./src/bibliography.bib}

\begin{document}

%% use roman page numbering for abstract and acknowledgements
\pagenumbering{roman}

%% title page
\begin{titlepage}
\begin{center}
    \vspace{0.1\textheight}
    {\Large Undergraduate Thesis 2022} \\
    \vspace{0.05\textheight}
    \includegraphics[width=48truemm]{resources/0_title/waseda_logo.png} \\
    \vspace{0.05\textheight}
    \textbf{\huge 'Campus' as 'Canvas': General Approach to Revitalize a Place with Location-based Augmented Reality and Co-creation
    } \\
    \vfill
    {\Large Supervisor \hspace{0.02\textwidth} Nakajima Tatsuo} \\
    {\Large Area of Study \hspace{0.02\textwidth} Computer Science} \\
    \vspace{0.05\textheight}
    {\Large 
        Waseda University \\
        School of Fundamental Science and Engineering \\
        Department of Computer Science \\}
    \vspace{0.05\textheight}
    {\Large 1W17BG08-2 Hu Yong-Hao \\}
    \vspace{0.05\textheight}
    {submitted on 2022.01.29}
\end{center}
\end{titlepage}

%% abstract
\pagebreak
\hspace{0pt}
\vfill % magic commands to vertically center the content
    \begin{center}
    Abstract
    \end{center}
Pandemic and digitalization results in an increase of places or facilities becoming desolate or abandoned.
Since both location-based AR and Co-creation have been proved to have influence on users' motivation and the places they are implemented,
for revitalization of places in general, we composed a framework consisting of location-based AR and Co-creation together.
We developed a prototype with components described in the framework, and then we conducted an experiment where participants used the prototype in Nishi-Waseda campus of Waseda University.
Positive results of location-based AR and Co-creation on participants' motivation to access the campus and images of the campus in their mind were observed,
which indicates the feasibility for our framework to revitalize the campus. We then discussed its generalization to other public facilities.
Limitations in this study include restrictions in AR experience and ambiguous evaluation for certain items, and we also suggested future directions to improve the limitations as well as further evaluate the generalization of the proposed framework.
\vfill
\pagebreak

%% acknowledgement
% \pagebreak
% \hspace{0pt}
% \vfill 
%     \begin{center}
%     Acknowledgements
%     \end{center}
% This is my acknowledgements...

% \vfill
% \pagebreak

%% use normal page numbering for the main body
\pagenumbering{arabic}

%% table of contents 
\tableofcontents
\listoffigures 
\listoftables

%% main body
\clearpage
% % \chapter{Notations}
% Sample notations \autoref{ta:notations}

% % table of notation
% \setlength{\tabcolsep}{6pt}
% \begin{table} 
% \begin{center} 
% \caption{Mathematical notations}
% \label{ta:notations}
% {\normalsize
% \begin{tabular}{l c}
% \toprule
% Symbol & Meaning \\
% \midrule
%     $\alpha$ & learning rate\\
%     $\gamma$ & discount factor \\
%     $S, s$ &   state \\
%     $A, a$ &   action \\
%     $R, r$ &   reward \\
%     $\tau$ &    a trajectory / an episode \\
%     $G$ &   return \\
%     $t$ &   a discrete time step \\
%     $G_t$ & return at time step t \\
%     $T$ &   final time step of an episode \\
%     $\pi$     & policy \\
%     $\pi_\theta$ & parametrized policy with parameter \theta \\
%     $\pi(s)$ & the action distribution given state $s$ under policy $\pi$ \\
%     $\pi(a|s)$ & probability of action $a$ given state $s$ under policy $\pi$ \\
%     $\E$ & expectation \\
%     $\E_\pi$ & expectation under policy $\pi$ \\
%     $v(s)$ & state value of state $S$ \\
%     $v_\pi(s)$ & state value of state $S$ under policy $\pi$\\
%     $q(s, a)$ & action value of action $a$ on state $s$ \\
%     $q_pi(s, a)$ & action value of action $a$ on state $s$ under policy $\pi$\\
%     $\sigma$ & activation function \\
% \bottomrule
% \end{tabular}
% } \end{center} \end{table}
% \setlength{\tabcolsep}{6pt}





\chapter{Introduction}
% Topic

\section{Motivations}
% Current situations
% Why there needs further research

As the pandemic of COVID-19 spreading throughout the world since 2020,
people were forced or encouraged to stay home and restricted from accessing public places,
including tourist attractions, shops, workplaces, schools, etc.
Humans' freedom in physical space is restricted, which accelerate the progress of digitalization.
Not only entertainment but more and more economic and even academic activities are moving online.
As the pandemic slowing down recently, despite the resumption of some physical activities,
there are places or facilities remaining unused or abandoned due to financial problems,
amount of users not recovered, digitalization of activities, and so on.

Removing the idle places or facilities is a choice, but if it is possible to give them new values or change people's image of them,
they can play different roles and keep contributing the society or enrich the environment.
In fact, the concept 'Regional Revitalization', which referes to the attempts to vitalize rural towns where population is falling,
by making use of local speciality combined with new ideas to develop new and unique industries such as tourism, has been applied around Japan recently.
Among cases of Regional Revitalization, some of them adopt location-based Augmented Reality to help enrich the space.
Location-based Augmented Reality is defined as Augmented Reality that utilize geographical information to display contents corresponding to a physical location.
It has already used in not merely entertainment, where Pokemon GO is a famous example,
but also implemented in tourism and education, which implies its versatility and practicability.
With the application of location-based Augmented Reality and the reference of Regional Revitalization,
transformation of an idle place or facility without physical reconstruction seems to be feasible.

% BUT not all places are tourist attractions; each place has its unique features and original functionalities; customizing contents for each place is difficult
  % -> Common characteristics: People!
  % -> Enable people to create new values / images for the place (co-creation)
% BUT how to encourage people?
  % -> User-user interaction!

\section{Objectives and importance}
% RQ
【RQ】Location-based AR+ユーザ共創のコンテンツが:
ある場所をもっと魅力的にする?
ある場所がユーザにとっての意義が変わる?
ユーザ同士の交流ができている?

% goals
キャンパスの元気を取り戻す and answer the RQs
propose a general model to vitalize an arbitrary place

% hypotheses
significant, positive effect

% importance
* (Preliminary studies of) Local transformation in general, instead of only tourism or education goals
* Use ‘people’ to comprise the contents, instead of considering specific characteristics of each location
* Prove a possibility, instead of focusing on detailed improvement of technology
    * Future work: combined with improved technology

\section{Overview of this paper}
% order of sections in this paper
% outline the methodology

\section{Sample section}

Sample template \cite{alphago}

\begin{figure}
  \centering
  \includegraphics[width=\columnwidth]{resources/1_intro/vgc2019.png}
    \caption{Screenshot of the Grand Finals of the Pokemon Video Game Championships 2019 held in Washinton D.C.}
\end{figure}


\chapter{Backgrounds} \label{ch:2}
% Explain concepts required to understand this paper, with including references to existing works that introduced the concepts

\section{Pandemic's impact}

\section{Local transformation}

\section{Location-based Augmented Reality}

\section{Co-creation}
\chapter{Related Works} \label{ch:3}
% summarize existing related / similar works, and discuss how my paper differs from them

\section{Location-based AR's effect on users’ image of a place}

\section{Location-based AR's effect on users’ motivation of a place}

\section{Co-creation's effect on users’ motivation to access a service}

\section{User-user interaction's effect on users’ engagement of a service}

\section{Local transformation by graffiti}

\section{More examples of location-based service with co-creation}

% add figures from the VC paper
\chapter{Methodology}\label{ch:4}

\section{Proposed Framework}
Reviewing a variety of regional revitalization cases, we sketched a diagram at Figure 4.1 to summarize their common mechanism.
In a common case of regional revitalization, the authority makes use of local specialties and applies new ideas with technology to improve existing industry or establish a new one,
usually a tourism business, which succeeds to attract more people to visit the place and activate local economy.

We also sketched a diagram at Figure 4.2 to describe a common mechanism of regional revitalization that implements location-based AR.
In such cases, the authority applies new ideas on local features to compose unique contents for a location-based AR service,
which motivates people to access the place more, resulting in an improvement in local economy.
Despite that the contents are in digital form or accessible online, the system's location-based characteristics still make it to encourage visitors to access physically.

\begin{figure}
  \begin{minipage}{0.48\textwidth}
    \centering
    \includegraphics[width=0.9\linewidth]{resources/4_methodology/common_vitalization.png}
      \caption{Common framework of Regional Revitalization}
  \end{minipage}\hfill
  \begin{minipage}{0.48\textwidth}
    \centering
    \includegraphics[width=0.9\linewidth]{resources/4_methodology/revitalization_with_AR.png}
      \caption{Framework of Regional Revitalization with location-based AR}
  \end{minipage}
\end{figure}

For places like public facilities where there is a lack of local features usable to attract visitors,
we presented a framework, sketched in Figure 4.3, that adopts a common characteristic of the places: users.
In our assumption, by enabling users to engage in co-creation of contents, which can be conducted digitally with low costs in a location-based AR system,
we anticipate that the problem of lacking usable resources becomes solvable.
Besides the issue of content creation, From Section 3.1, 3.2 and 3.3 we understand the influence on users' motivation and images about a place by both location-based AR and co-creation,
which are both included in our framework.
We also introduce a socializing mechanism to encourage users to participate in the co-creation process.
From Section 3.4 we understand that interaction between users improves people's engagement with a co-creation activity.

For this framework we proposed, we developed a prototype according to the idea of the framework, and later we examined the proposed framework with an experiment with the prototype.

\begin{figure}
  \centering
  \includegraphics[width=0.8\columnwidth]{resources/4_methodology/revitalization_with_AR_and_cocreation.png}
    \caption{Proposed framework: Revitalization with location-based AR and Co-creation}
\end{figure}


\section{Prototype}

\chapter{Experiment and Results}\label{ch:5}

\section{Evaluation}

\subsection{Evaluation Targets}

In the proposed framework, there are several targets we had to evaluate in order to answer our research questions.
The following list explains the evaluation targets, and Figure 5.1 indicates where the targets are located in our proposed framework.

\begin{quote}
  \begin{itemize}
    \item T1: Motivation to access the campus by location-based AR contents
    \item T2: Motivation to access the campus by Co-creation process
    \item T3: Motivation to access the campus by interaction among users
    \item T4: Changes in image of the campus by location-based AR contents
    \item T5: Changes in image of the campus by Co-creation process
    \item T6: Changes in image of the campus by interaction among users
    \item T7: Motivation to access the campus by the whole framework
    \item T8: Changes in image of the campus by the whole framework
  \end{itemize}
\end{quote}

Targets T1, T2, T4, T5, T7, T8 correspond to the first and second research questions:
the influence of location-based AR and Co-creation on both motivation and changes in image.
T1 and T4 correspond to location-based AR, T2 and T5 correspond to Co-creation, and T7 and T8 correspond to the combined influence.
Target T3 corresponds to the third research question about user-user interaction's effect on motivation.
Target T6 does not correspond to any research question we set since we did not planned to see user-user interaction's effect on image changing,
but we decided to include it to see if we can have unexpected findings.

\begin{figure}[ht]
  \centering
  \includegraphics[width=0.8\columnwidth]{resources/5_experiment_and_results/proposed_framework_with_evaluation_targets.png}
    \caption{Proposed framework and targets to evaluate}
\end{figure}

\subsection{Evaluation of Motivation}

To evaluate targets about motivation, including T1, T2, T3 and T7, we adopted questions from Situational Motivation Scale (SIMS) \cite{guay_vallerand_blanchard_2000} for measurement.
SIMS contains four categories of motivation: 'Intrinsic motivation', 'Extrinsic motivation', and 'Amotivation', while in this study we specifically adopted 'Intrinsic motivation (IM)' and 'Amotivation (AM)'.
'Extrinsic motivation', including 'Identified regulation (IR)' and 'External regulation (ER)' are excluded since what we wanted to measure is the motivation induced by components in the proposed framwork, instead of following any instruction, obiligation, or any other external factors.
Questions we adopted from SIMS are listed in Appendix A.
Here we adopted 6 point scales to avoid ambiguous responses (where users keep choosing the middle item).

Cahyono et al. adopted SIMS with the use of Self-Determination Index (SDI) for scoring, which is calculated by the formula below:
\[ SDI = (2 * IM) + IR - ER - (2 * AM) \]
The higher the value of SDI, the more intrinsically motivated a person is \cite{cahyono_ludwig_2017}.
However, since we excluded IR and ER in this study, we conducted the scoring with:

\[ IM - AM \]

and composed a hypothesis for the scoring of motivation:

\begin{quote}
  Value of IM - AM measured with SIMS are positive for location-based AR, Co-creation, user-user interaction, and the combination of them.
\end{quote}

% \begin{quote}
%   \begin{itemize}
%     \item H1: For T1, the value of IM - AM measured with SIMS is positive.
%     \item H2: For T2, the value of IM - AM measured with SIMS is positive.
%     \item H3: For T3, the value of IM - AM measured with SIMS is positive.
%     \item H4: For T7, the value of IM - AM measured with SIMS is positive.
%   \end{itemize}
% \end{quote}

Besides the scales, we also prepared questions for free comments about motivation.

\subsection{Evaluation of Changes in Image}

To evaluate targets about changes in image of the campus, including T4, T5, T6 and T8,
we prepared a question with 5 point scale, described as follows:

\begin{quote}
  Does the image of the campus in your mind changed?
  \begin{enumerate}
    \item Not at all
    \item Only a little
    \item Somehow changed
    \item Changed a lot
    \item Completely changed
  \end{enumerate}
\end{quote}

Besides the scales, we also prepared questions for free comments about changes in image of the campus.

\subsection{Questionnaires}

Then we designed 4 questionnaires prepared for participants in an experiment conducted later (explanation in section 5.2). Each questionnaire corresponds to a factor listed as follows:

\begin{quote}
  Factors
  \begin{enumerate}
    \item Viewing location-based AR contents, the graffiti, in the campus
    \item Creating location-based AR contents, the graffiti, in the campus
    \item Interactions with other users
    \item Overall experience of using the prototype
  \end{enumerate}
\end{quote}

In each questionnaire, we asked questions about how the experience of the factor during the experiment affected one's motivation to access the campus, with questions introduced in Section 5.1.2, as well as changes in image of the campus in one's mind, with questions introduced in Section 5.1.3.
For example, Questionnaire 1 includes questions about the motivation and changes in image of the campus influenced by the experience of Factor 1: viewing location-based AR contents in the campus.

Results from Questionnaire 1 correspond to the evaluation of T1 and T4, Questionnaire 2 to T2 and T5, Questionnaire 3 to T3 and T6, and finally Questionnaire 4 to T7 and T8.
% In Questionnaire 1, 2 and 4, we also asked questions about participants' feeling of presence to check its relevance to the questionnaire's factor.
In Questionnaire 3, we also included questions about awareness of other users' existence and interaction with them, in order to confirm that the socializing mechanism functions effectively in the prototype.
Eventually, in each questionnaire, we also asked whether a participant, after attending the experiment, prefers our location-based AR prototype or a similar one without location-based features and AR effect but usable at home,
in order to make clear of the importance of location-based AR.

\section{Experiment}

At first, we conducted a preliminary survey with 3 participants trying the prototype in Waseda University Nishi-Waseda Campus for one week.
3 participants gave us positive responses about their motivation to access campus after experiencing the prototype.
We also improved the app based on their feedbacks, such as adding features that allow users to review/edit/delete their own graffiti.
The experinemt lasted for 2 weeks. 14 males and 2 females participated,
and they are asked to use the prototype freely in the same campus at least twice a week.
Before the experiment, we asked participants about their frequencies of accessing the campus and the images of campus in their mind before and after the pandemic started spreading,
in order to understand how much impact the pandemic brought on each participant.
Instruction of using the prototype was also distributed before the experiment.
2 weeks later, after the experiment finished, participants were required to answer the questionnaires introduced in Section 5.1.

We also conducted a control experiment, with 3 males and 1 females participating in playing a similar prototype without location-based features and AR effect but usable at home for a week.
Then we asked them to fill in the same questionnaires.

\section{Results}
\subsection{Motivations}

Table 5.1 to 5.4 shows the evaluation results of motivation to access campus from Questionnaire 1 to 4 respectively.
All of them had positive values of either the average or the median of IM - AM, which verifies our hypothesis for the scoring of motivation.
In addition, among results from Table 5.1, 5.2 and 5.3, although they did not vary much, results by Factor 1 (mean of IM - AM: 1.6406, median of IM - AM: 1.6250) are the lowest,
and results by Factor 2 (mean of IM - AM: 1.9667, median of IM - AM: 2.0000) are the highest.
% This indicates that participation in Co-creation motivates the most, and viewing location-based AR contents motivates the least.
Notably, standard deviation of IM - AM by Factor 3 (value: 2.0743) is the highest and the only one higher than 2.0000.
% , indicating the variability in responses with regard to motivation influenced by interaction with other users.
Last but not least, in Table 5.4, results by Factor 4 (mean of IM - AM: 2.0156, median of IM - AM: 2.6250) had higher values than those by Factor 1 to 3.
% , indicating that a combination of location-based AR contents, Co-creation and interaction has a better effect on improving motivation.

\begin{table}[h]
  \caption{Motivation to access campus influenced by Factor 1: viewing location-based AR contents}
    \label{table:1}
  \begin{tabular}{l || R{4cm} | R{3cm} | R{2.5cm}}
    \hline
    \rowcolor{lightgray}
          & \multicolumn{1}{c |}{Intrinsic motivation (IM)} & \multicolumn{1}{c |}{Amotivation (AM)} & \multicolumn{1}{c}{IM - AM}  \\
    \hline
    N      & 16     & 16     & 16      \\
    Mean   & 4.3906 & 2.7500 & 1.6406  \\
    Median & 4.3750 & 3.0000 & 1.6250  \\
    Min    & 3.0000 & 1.0000 & -1.5000 \\
    Max    & 6.0000 & 4.5000 & 5.0000  \\
    SD     & 0.7636 & 1.0124 & 1.5916  \\
    \hline
  \end{tabular}
\end{table}

\begin{table}[h]
  \caption{Motivation to access campus influenced by Factor 2: participation in Co-creation}
    \label{table:2}
  \begin{tabular}{l || R{4cm} | R{3cm} | R{2.5cm}}
    \hline
    \rowcolor{lightgray}
          & \multicolumn{1}{c |}{Intrinsic motivation (IM)} & \multicolumn{1}{c |}{Amotivation (AM)} & \multicolumn{1}{c}{IM - AM}  \\
    \hline
    N      & 15     & 15     & 15      \\
    Mean   & 4.4833 & 2.5167 & 1.9667  \\
    Median & 4.2500 & 2.5000 & 2.0000  \\
    Min    & 3.0000 & 1.0000 & -1.5000 \\
    Max    & 6.0000 & 4.5000 & 5.0000  \\
    SD     & 0.8044 & 1.0021 & 1.6767  \\
    \hline
  \end{tabular}
\end{table}

\begin{table}[h]
  \caption{Motivation to access campus influenced by Factor 3: interaction with other users}
    \label{table:3}
  \begin{tabular}{l || R{4cm} | R{3cm} | R{2.5cm}}
    \hline
    \rowcolor{lightgray}
          & \multicolumn{1}{c |}{Intrinsic motivation (IM)} & \multicolumn{1}{c |}{Amotivation (AM)} & \multicolumn{1}{c}{IM - AM}  \\
    \hline
    N      & 15     & 15     & 15      \\
    Mean   & 4.3833 & 2.5833 & 1.7200  \\
    Median & 4.7500 & 2.0000 & 1.8000  \\
    Min    & 2.0000 & 1.0000 & -2.0000 \\
    Max    & 6.0000 & 5.0000 & 5.0000  \\
    SD     & 1.1135 & 1.1286 & 2.0743  \\
    \hline
  \end{tabular}
  \end{table}
  
  \begin{table}[h]
    \caption{Motivation to access campus influenced by Factor 4: overall experience of the prototype}
      \label{table:4}
    \begin{tabular}{l || R{4cm} | R{3cm} | R{2.5cm}}
    \hline
    \rowcolor{lightgray}
          & \multicolumn{1}{c |}{Intrinsic motivation (IM)} & \multicolumn{1}{c |}{Amotivation (AM)} & \multicolumn{1}{c}{IM - AM}  \\
    \hline
    N      & 16     & 16     & 16      \\
    Mean   & 4.5469 & 2.5313 & 2.0156  \\
    Median & 4.6250 & 2.2500 & 2.6250  \\
    Min    & 3.0000 & 1.0000 & -2.0000 \\
    Max    & 6.0000 & 5.0000 & 5.0000  \\
    SD     & 0.8328 & 1.0950 & 1.8108  \\
    \hline
  \end{tabular}
\end{table}

We also collected free comments about motivation in each questionnaire. In Questionnaire 1 we received responses for IM, including 'I become curious about other people's graffities and their comments on my drawings',
'I feel more creative and fun by sharing works with others', 'It is fun to secretly see my friends' drawings', 'Sometimes I felt connected to other students',
which we considered are related to interaction with users as well.
In Questionnaire 3 we received responses with obviously opposite attitudes: one states 'I feel like I can make friends with this' for IM and another one answered 'Interaction with user can be done online too in my opinion' for AM.
% This corresponds to the variability with regard to motivation influenced by interaction with other users, which we observed from the evaluation results of Table 5.3 and mentioned in the last paragraph.
In Questionnaire 4 we received a response for AM that states '... as long as the social distance (due to the pandemic) exists, there may be many constraints on AR since it mainly bases on reality',
pointing out the limitation of AR under pandemic circumstances.

\subsection{Image of the campus}

Table 5.5 shows the evaluation results of changes in image of the campus from Questionnaire 1 to 4.
Each factor resulted in a mean value between 2 (Changed a little) and 3 (Somehow changed) as well as a median value equal to 3 (Somehow changed).
Among Factor 1, 2 and 3, Factor 3 resulted in the highest mean value (2.8667), and Factor 1 resulted in the lowest one (2.6875),
% indicating that interaction among users changed the image the most, and viewing location-based AR contents changed the least.
while Factor 3 had the highest standard deviation (0.8338).
% , indicating the variability of responses by interaction among users.
In addition, results by Factor 4 (mean of IM - AM: 2.0156, median of IM - AM: 2.6250) had higher values than those by Factor 1 to 3.
% indicating that a combination of location-based AR contents, Co-creation and interaction has a better effect on changing image.

We also collected free responses, listed in Appendix B. Most of the responses express positive changes, such as "I used to feel that the campus was quiet and there was little interaction between people, but through this content, I learned that I could interact with strangers, and my image of the campus became more sociable.",
"I hadn't had a chance to take a good look at the campus, so it was refreshing.", "I started to think sometimes about what things on campus could look like.", "I developed a common feeling that we were all students at the same university.",
while there are also negative opinions, such as "The campus became a little more fun, but it wouldn't have changed my overall image.", "To me it was just an application on phone where I can draw and see others' works", "I thought that since the interaction was with people, it had little impact on the image of the campus.".

\begin{table}[h]
  \caption{Changes in image of the campus by different factors, scaled from 1 (Not at all) to 5 (Completely changed)}
    \label{table:5}
  \begin{tabular}{l || R{3.5cm} | R{2.5cm} | R{2cm} | R{2cm}}
    \hline
    \rowcolor{lightgray}
          & \multicolumn{1}{C{3.5cm} |}{1. View location- \newline based AR contents} & \multicolumn{1}{C{2.5cm} |}{2. Participate \newline in Co-creation} & \multicolumn{1}{C{2cm} |}{3. User-user \newline interaction} & \multicolumn{1}{C{2cm}}{4. Overall \newline experience} \\
    \hline
    N      & 16     & 15     & 15     & 16     \\
    Mean   & 2.6875 & 2.7333 & 2.8667 & 2.8750 \\
    Median & 3.0000 & 3.0000 & 3.0000 & 3.0000 \\
    Min    & 2.0000 & 2.0000 & 2.0000 & 2.0000 \\
    Max    & 4.0000 & 4.0000 & 4.0000 & 4.0000 \\
    SD     & 0.6021 & 0.5936 & 0.8338 & 0.8062 \\
    \hline
  \end{tabular}
\end{table}

\subsection{User-user Interaction}

In Questionnaire 3, we also evaluated participants' sense of other users' existence and interaction,
and the results are displayed in Table 5.6. Among 1 (Disagree) to 6 (Agree), mean and median values of both existence and interaction are more than or equal 4.
% indicating that the protoype succeeded to make users feel others' existence and build interaction between them.
The free responses we collected contain "I went to use this system with another participant. We were playing a game of guessing which graffiti each other had drawn" and "I used it with my classmates, and we talked about what we were drawing".
% which described real cases of interaction among users with our framework.

\begin{table}[h]
  \caption{Sense of other users' existence and interaction, scaled from 1 (Disagree) to 6 (Agree)}
    \label{table:6}
  \begin{tabular}{l || R{5cm} | R{5.5cm}}
    \hline
    \rowcolor{lightgray}
          & \multicolumn{1}{C{5cm} |}{I felt existence \newline of other users} & \multicolumn{1}{C{5.5cm}}{It felt like I am \newline interacting with other users} \\
    \hline
    N      & 15     & 15     \\
    Mean   & 4.5333 & 4.0000 \\
    Median & 5.0000 & 4.0000 \\
    Min    & 3.0000 & 2.0000 \\
    Max    & 6.0000 & 6.0000 \\
    SD     & 0.9155 & 1.3093 \\
    \hline
  \end{tabular}
\end{table}

% \subsection{Revelance of Feeling of Presence}

\subsection{Comparison with situation without location-based AR}

Table 5.7 shows the evaluation results of participants' preference between prototype at campus or situation at home by different factors.
For each factor, the mean value is between 4 and 5, and the median value equals 5 or 6. This indicates a higher preference for our prototype where location-based AR features is implemented.
We also collected free responses about the preference, with some of which listed in Appendix C.
Responses that prefer the case at campus mainly express that the prototype at campus with location-based AR creates more fun, enables exploration of different perspectives and sense of realism, or provides chances to meet other users in reality.
Responses that prefer the case at home mainly point out difficulties to use at campus due to the environment or people passed by, question the necessary of interacting with other users in reality, or express the weariness of physically traveling around the campus.

\begin{table}[h]
  \caption{Preference between prototype at campus or situation at home by different factors, scaled from 1 (At home) from 7 (At campus)}
    \label{table:7}
  \begin{tabular}{l || R{3.5cm} | R{2.5cm} | R{2cm} | R{2cm}}
    \hline
    \rowcolor{lightgray}
          & \multicolumn{1}{C{3.5cm} |}{1. Viewing location- \newline based AR contents} & \multicolumn{1}{C{2.5cm} |}{2. Participation \newline in Co-creation} & \multicolumn{1}{C{2cm} |}{3. User-user \newline interaction} & \multicolumn{1}{C{2cm}}{4. Overall \newline experience} \\
    \hline
    N      & 16     & 15     & 15     & 16     \\
    Mean   & 4.8125 & 4.8667 & 4.6000 & 4.7500 \\
    Median & 5.0000 & 6.0000 & 5.0000 & 6.0000 \\
    Min    & 1.0000 & 1.0000 & 1.0000 & 1.0000 \\
    Max    & 7.0000 & 7.0000 & 7.0000 & 7.0000 \\
    SD     & 1.6419 & 1.8074 & 1.6388 & 1.8074 \\
    \hline
  \end{tabular}
\end{table}

In the control experiment where participants used another prototype without location-based AR features, evaluation of motivation shows negative evaluation values,
and evaluation results of image changing of campus are less significant than the results from the formal experiment.
Results of motivation and image changing are listed in Table 5.8 and 5.9 respectively.

% table: mean of each IM & AM & IM-AM, mean of each image changing

About preference, we received free responses including "Doing this at home is more like drawing some thing on my picture. I don't feel fun drawing on it.", "I think that's the reason why AR is needed...? To explore somewhere that you can't actually visit" , 
"AR seems to more interesting. There are many other interesting things to do at home." and "I personally prefer directly interacting with the users (person to person)", 
which show more interests in a situation with location-based AR than their non-AR experience in the control experiment.

\chapter{Discussion}\label{ch:6}

\section{Motivations}

\section{Image of Campus}

\section{User-user Interaction}

\chapter{Conclusion}\label{ch:7}

\section{Conclusion}

\section{Limitations}

\section{Future Works}


%% bib
\printbibliography

%% appendix
\appendix
\chapter*{APPENDIX A - Questions adopted from SIMS}

\begin{table}[h]
\begin{center}
\caption{Questions adopted from SIMS}\label{table:8}
\begin{tabular}{C{2cm} | l | C{1.9cm}}
    \hline
    \rowcolor{lightgray}
        \multicolumn{1}{C{2cm}}{Motivation \newline type} & \multicolumn{1}{c}{Questions} & \multicolumn{1}{C{1.9cm}}{Scale} \\
    \hline
    \multirow{4}{2cm}{Intrinsic motivation (IM)} & Because I think it is interesting. & \multirow{4}{1.9cm}{1 Disagree - 6 Agree} \\
        & Because I think it is pleasant. & \\
        & Because this is fun. & \\
        & Because I feel good when experiencing it. & \\
    \hline
    \multirow{4}{2cm}{Amotivation (AM)} & Personally I don't see any good reason to do it. & \multirow{4}{1.9cm}{1 Disagree - 6 Agree} \\
        & I'm not sure if it is worth it. & \\
        & I don't see what it brings me. & \\
        & I'm not sure it is a good thing to do. & \\
    \hline
\end{tabular}
\end{center} 
\end{table}


\chapter*{APPENDIX B - Free responses of changes in image of the campus}

\begin{table}[h]
\begin{center}
    \caption{Free responses of changes in image of the campus by viewing location-based AR contents}\label{table:9}
    \begin{tabular}{L{\textwidth}}
        \hline
        \rowcolor{lightgray}
          \multicolumn{1}{c}{Free responses} \\
        \hline
          {
            \begin{itemize}
              \item When I think other people draw at the place in campus, I want to check their artworks.
              \item I started to see places which I don't normally see.
              \item Many of the graffiti made things on campus look like something else, so the next time I saw it, I could think of the graffiti.
              \item I hadn't had a chance to take a good look at the campus, so it was refreshing.
              \item I couldn't draw pictures well, so I felt that there was a lack of reality (a sense of match with the real world).
              \item I tended to feel like I'm the only one in the campus, but when I think that everyone came to the university and looked at this remote place through the app, it brings something to my heart. It makes me feel closer to them.
              \item I used to feel that the campus was quiet and there was little interaction between people, but through this content, I learned that I could interact with strangers, and my image of the campus became more sociable.
              \item The campus became a little more fun, but it wouldn't have changed my overall image.
              \item To me it was just an application on phone where I can draw and see others' works
            \end{itemize}
          } \\
        \hline
    \end{tabular}
\end{center} 
\end{table}

\begin{table}[h]
    \begin{center}
      \caption{Free responses of changes in image of the campus by participation in co-creation}\label{table:10}
      \begin{tabular}{L{\textwidth}}
        \hline
        \rowcolor{lightgray}
          \multicolumn{1}{c}{Free responses} \\
        \hline
          {
            \begin{itemize}
              \item I began to look for a place where I could paint.
              \item I started looking at places I don't normally look.
              \item I started to think sometimes about what things on campus could look like.
              \item It's like we're all looking at the same place.
              \item I felt as if even the scenery I usually see is art from certain angles.
              \item I used to have an image of the campus as "less social", but this content has changed my image to "more sociable".
            \end{itemize}
          } \\
        \hline
    \end{tabular}
\end{center} 
\end{table}

\begin{table}[h]
    \begin{center}
      \caption{Free responses of changes in image of the campus by interaction between users}\label{table:11}
      \begin{tabular}{L{\textwidth}}
        \hline
        \rowcolor{lightgray}
          \multicolumn{1}{c}{Free responses} \\
        \hline
          {
            \begin{itemize}
              \item I went to more places when there were other users.
              \item We started to talk about the building and other things.
              \item I thought that since the interaction was with people, it had little impact on the image of the campus.
              \item I developed a common feeling that we were all students at the same university.
              \item There were pictures that made me wonder if that was the way to think.
            \end{itemize}
          } \\
        \hline
    \end{tabular}
\end{center} 
\end{table}

\begin{table}[h]
    \begin{center}
      \caption{Free responses of changes in image of the campus by overall experience}\label{table:12}
      \begin{tabular}{L{\textwidth}}
        \hline
        \rowcolor{lightgray}
          \multicolumn{1}{c}{Free responses} \\
        \hline
          {
            \begin{itemize}
              \item Overall, I started to pay more attention to the campus.
              \item I started to look at things on campus as different things, and remembered that other people had looked at things like this
              \item The campus had a gloomy image, but it changed to a sociable one.
            \end{itemize}
          } \\
        \hline
    \end{tabular}
\end{center} 
\end{table}

\chapter*{APPENDIX C - Free responses of preference between prototype at campus or situation at home}

\begin{table}[h]
  \begin{center}
    \caption{Example responses of preference between prototype at campus or situation at home by viewing location-based AR contents}\label{table:13}
    \begin{tabular}{C{2.5cm} | L{10cm}}
      \hline
      \rowcolor{lightgray}
      \multicolumn{1}{C{2.5cm}}{Preference} & \multicolumn{1}{c}{Example responses} \\
      \hline
        At campus & {
          \begin{itemize}
            \item It's easier for your brain to connect the actual place with the place in the graffiti.
            \item It gives a strong sense of actual experience and interaction.
            \item ... sitting at home and watching graffiti with pictures of the campus in the background makes it obvious that you are outside of that world. I believe that the experience you get will be completely different.
          \end{itemize}
        } \\
        \hline
        At home & {
          \begin{itemize}
            \item I couldn't help but notice the eyes around me.
            \item It's exhausting to travel around to check out the graffiti.
            \item I am more an indoor type person
          \end{itemize}
        } \\
      \hline
  \end{tabular}
\end{center} 
\end{table}

\begin{table}[h]
  \begin{center}
    \caption{Example responses of preference between prototype at campus or situation at home by participation in co-creation}\label{table:14}
    \begin{tabular}{C{2.5cm} | L{10cm}}
      \hline
      \rowcolor{lightgray}
      \multicolumn{1}{C{2.5cm}}{Preference} & \multicolumn{1}{c}{Example responses} \\
      \hline
        At campus & {
          \begin{itemize}
            \item It is more fun to draw on the spot.
            \item In the case of drawing at home, I didn't have as much freedom to choose my point of view as I did on campus, so it would be more interesting to draw on campus to explore different perspectives.
            \item I feel that it is important to draw while actually seeing buildings and other structures.
          \end{itemize}
        } \\
        \hline
        At home & {
          \begin{itemize}
            \item I can draw more calmly at home.
            \item I can paint without worrying about passersby.
          \end{itemize}
        } \\
      \hline
  \end{tabular}
\end{center} 
\end{table}

\begin{table}[h]
  \begin{center}
    \caption{Example responses of preference between prototype at campus or situation at home by interaction between users}\label{table:15}
    \begin{tabular}{C{2.5cm} | L{10cm}}
      \hline
      \rowcolor{lightgray}
      \multicolumn{1}{C{2.5cm}}{Preference} & \multicolumn{1}{c}{Example responses} \\
      \hline
        At campus & {
          \begin{itemize}
            \item I felt like we should be interacting in a real place.
            \item I would still prefer the real world interaction with other users at campus since it's easier to understand people's feelings and have a conversation.
            \item If you don't experience it at the place, you won't feel the realism and it won't be as interesting.
          \end{itemize}
        } \\
        \hline
        At home & {
          \begin{itemize}
            \item I thought that if the main purpose is to interact with people, there is no need to be on a campus.
            \item I felt that if we were just going to doodle together, we could do it online, like an online drawing chat, because it's easy to do at the same time.
          \end{itemize}
        } \\
      \hline
  \end{tabular}
\end{center} 
\end{table}

\begin{table}[h]
  \begin{center}
    \caption{Example responses of preference between prototype at campus or situation at home by overall experience}\label{table:16}
    \begin{tabular}{C{2.5cm} | L{10cm}}
      \hline
      \rowcolor{lightgray}
      \multicolumn{1}{C{2.5cm}}{Preference} & \multicolumn{1}{c}{Example responses} \\
      \hline
        At campus & {
          \begin{itemize}
            \item I could do it face-to-face with other users, and I could only encounter the artwork when I went there.
            \item ... using at campus is more interesting because you can choose your point of view more freely.
            \item If you don't experience it at the place, you won't get the sense of realism and it won't be as interesting. If you are there, you will be able to observe the actual situation more closely, which will give you more ideas for your doodles.
          \end{itemize}
        } \\
        \hline
        At home & {
          \begin{itemize}
            \item I can doodle without worrying about what others passed by.
            \item It's hard to concentrate when using it outside by yourself due to various factors such as temperature and people passed by.
          \end{itemize}
        } \\
      \hline
  \end{tabular}
\end{center} 
\end{table}


\addcontentsline{toc}{chapter}{APPENDIX A}  % unnumbered chapters are not added to TOC by default so adding it automatically



\end{document}

